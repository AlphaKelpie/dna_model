\documentclass[a4paper,14pt]{article}
\pdfpagewidth
\paperwidth
\pdfpageheight
\paperheight
\usepackage[english]{babel}
\usepackage[utf8]{inputenc}
\usepackage[babel]{csquotes}    %per biblatex
\usepackage[sorting=none,]	%mettere per ordinare secondo l'ordine di chiamata
{biblatex}   %bibliografia
\usepackage{epsfig}
\usepackage{fancyhdr} 
\usepackage{amsmath,amssymb}
\usepackage{amscd} 
\usepackage[T1]{fontenc} 
\usepackage[usenames,dvipsnames]{xcolor}
\usepackage{graphicx,color,listings}
\graphicspath{{../fig/}}	%percorso per le immagini
\usepackage{placeins}   %per \floatbarrier
\usepackage{hologo}
\frenchspacing
\usepackage{rotating}
\usepackage[bf,small]{caption}	%pacchetto caption
\captionsetup{tableposition=top,figureposition=bottom,font=small}
\usepackage{textcomp, gensymb}  %textcomp è per usare %, ° e altri simboli
\usepackage{hyperref}	%tutti i tipi di collegamento tranne vref
\hypersetup{colorlinks=true,	%attiva colori
 linkcolor=red,	%ref
 filecolor=black,
 urlcolor=blue,		%url
 citecolor=green,		%cite
 pdftitle={DNA model}	%per citare questo pdf?
}
\usepackage{multirow}
%\usepackage{multicol}  %per scrivere su più colonne
%\setlength{\columnsep}{10mm}	%spazio fra le colonne
\usepackage{picture}
\usepackage{booktabs}
%\usepackage{subfig}	%per subfigure all'interno dello stesso ambiente figure
\usepackage{subcaption}
\usepackage{float}  %per [H]
%\usepackage{circuitikz}    %circuiti
\usepackage[separate-uncertainty=true]{siunitx} %unità di misura
\usepackage{newlfont}
\usepackage{setspace}   %interlinee diverse
\usepackage[version=4]{mhchem}  %formule chimiche
\usepackage[english]{varioref}  %referenze che espicitano la pagina
\usepackage{geometry}	%margini: già se presente li cambia
% \geometry{
%   left=15mm,
%   top=20mm,
%   right=15mm,
%   bottom=20mm,
% }

\bibliography{bib}

\title{Chiral selection in wrapping, crossover, and braiding of DNA mediated by asymmetric bend-writhe elasticity}
\author{Tommaso Rondini}

\begin{document}
\maketitle

\section*{Abstract}

\section{Introduction}

\section{Model of Double-Stranded DNA}
\subsection{Coordinates of the DNA model}
The model of DNA in that study consists of two elastic chains intertwining around a central backbone.
The central backbone is a chain with the same number of element as the bases chains.
The position of the $i$ element is given by a three dimensional vector $\textbf{r}_i \ \left(i=1,\dots ,N\right)$.
The distance between adjacent nodal point is $b = 0.34$\si{\nm}, which is the distance between to neighbouring base pairs.
The first node is set at some coordinates (usually $(0,\ 0,\ 0)^T$); the z-axis is in the direction of the second node, the y-axis lies in the plain made by the first three nodes and the x-axis is perpendicular to the z and y axis, creating a right-hand system.
In this way, the position of the second node is $(0,\ 0,\ b)^T$ in the reference system of the fist node.
We assign a three-dimension orthonormal frame $F_i$ to each node (it is useful for set the position of the bases): in each $i$ system the z-axis is an extension to the link between the $i-1$ and the $i$ node, the y-axis lies in the plain formed by $\textbf{r}_{i-1},\ \textbf{r}_i$ and $\textbf{r}_{i+1}$, and $\hat{x}_i=\hat{y}_i\times\hat{z}_i$.
Now we see that $F_2=F_1$.
We introduce angles $\Theta_2$ that rotate the $\hat{x}_2$-axis so that $\hat{z}_3$ is parallel to the link between the third and the second node.
We have to define $\Theta_i$ angles with $i=2,\dots,N-1$.
The fourth frame $F_4$ y-axis lies in a different plane, so we have to introduce dihedral angles $\Phi_i\ (i=2,\dots,N-2)$ to rotate around $\hat{z}_3$.
So we have:
\begin{equation}\label{eq:c_coo}
\textbf{r}_i=\textbf{r}_{i-1}+\textbf{R}^{2}_{3}\left(\Phi_1,\Theta_2\right)\textbf{R}^{3}_{4}\left(\Phi_2,\Theta_3\right)\dots\textbf{R}^{i-1}_{i}\left(\Phi_{i-2},\Theta_{i-1}\right)\textbf{b}\qquad (i=3,\dots,N)
\end{equation}
where $\textbf{b}=(0,\ 0,\ b)^T$, $\Phi_1=0\deg$, and
\begin{equation}\label{eq:r_matrix}
\textbf{R}^{i-1}_{i}\left(\Phi_{i-2},\Theta_{i-1}\right)=
\begin{pmatrix}
\cos\Phi_{i-2} & -\sin\Phi_{i-2} & 0 \\
\sin\Phi_{i-2} & \cos\Phi_{i-2} & 0 \\
0 & 0 & 1 \\
\end{pmatrix}
\begin{pmatrix}
1 & 0 & 0 \\
0 & \cos\Theta_{i-1} & -\sin\Theta_{i-1} \\
0 & \sin\Theta_{i-1} & \cos\Theta_{i-1} \\
\end{pmatrix}
\end{equation}

The two chains of the model represent the two sugar-phosphate chains of DNA.
The bonds between adjacency sugar-phosphate elements are elastic, as harmonic springs).
Their positions are represented by $\textbf{P}_i$ and $\textbf{Q}_i$ respectively.
The distance between the backbone structure and the DNA chains is fixed to $\sigma_0=1.0$\si{\nm}, and p and q elements are linked with a rigid rod that is the hydrogen-bonded base pair.
We introduce $\psi_i\ (i=1,\dots,N)$ that is the internal twist angle characterizing the relative orientation of \textit{i-th} base pair with respect to \textit{i-th} frame attached to the central backbone; and $\omega_0$ is the common constant angle between the two vectors $\textbf{P}_i-\textbf{r}_i$ and $\textbf{Q}_i-\textbf{r}_i$.
So the position of $p_i$ and $q_i$ respect to $\textbf{r}_i$ are
\begin{equation}
\begin{split}
\textbf{p}_i & =\left(\sigma_0\cos\psi_i,\ \sigma_0\sin\psi_i,\ 0\right)^T \qquad (i=1,\dots,N) \\
\textbf{q}_i & =\left(\sigma_0\cos\left(\psi_i+\omega_o\right),\ \sigma_0\sin\left(\psi_i+\omega_0\right),\ 0\right)^T \qquad (i=1,\dots,N)
\end{split}
\end{equation}
Instead, their absolute positions are: $\textbf{P}_1=\textbf{p}_1$, $\textbf{Q}_1=\textbf{q}_1$, $\textbf{P}_2=\textbf{r}_2-\textbf{p}_2$, $\textbf{Q}_2=\textbf{r}_2-\textbf{q}_2$,
\begin{equation}
\textbf{P}_i=\textbf{r}-\textbf{R}^{2}_{3}\left(\Phi_1,\Theta_2\right)\textbf{R}^{3}_{4}\left(\Phi_2,\Theta_3\right)\dots\textbf{R}^{i-1}_{i}\left(\Phi_{i-2},\Theta_{i-1}\right)\textbf{p}_i\qquad (i=3,\dots,N)
\end{equation}
\begin{equation}
\textbf{Q}_i=\textbf{r}-\textbf{R}^{2}_{3}\left(\Phi_1,\Theta_2\right)\textbf{R}^{3}_{4}\left(\Phi_2,\Theta_3\right)\dots\textbf{R}^{i-1}_{i}\left(\Phi_{i-2},\Theta_{i-1}\right)\textbf{q}_i\qquad (i=3,\dots,N)
\end{equation}
Thus, when $\psi_i$ changes, the nodal points $\textbf{P}_i$ and $\textbf{Q}_i$ move together.
$\Theta_i$, $\Phi_i$ and $\psi_i$ are the conformation parameters of this model.
Since the central backbone was been created just to construct the coordinates and has not any direct influence on the configuration energy, from now on in the figures we will not show it.

\subsection{Equilibrium conformation}
For the equilibrium conformation they assume that the central backbone takes a straight conformation (e.g. $\Theta_i=0\deg\ (i=2,\dots,N-1)$ and $\Phi_i=0\deg\ (i=2,\dots,N-2)$.
They assume that the two sugar-phosphate chains complete one helical cycle per every $10$ nodal points as in B-form DNA~\cite{1}\cite{2}.
Thus, at the equilibrium conformation the angle between two neighbouring nodal points is $\psi_0=36\degree$ with respect to the central backbone when vertically projected onto the plane perpendicular to the central backbones.
As a result, the pitch of each helical turn of the double strand is \SI{3.4}{\nm}.

The internal twist angle $\psi_i$ is so set to
\begin{equation}
\psi_i=\left(i-1\right) \psi_0
\end{equation}
In order to make the width of major groove \SI{2.2}{\nm} and that of minor groove \SI{1.2}{\nm} at the equilibrium conformation~\cite{1}\cite{2} they assume that the constant angle $\omega_0$ to be
\begin{equation}
\omega_0=1.2\times\dfrac{360\deg}{3.4}\simeq127.06\degree
\end{equation}
Based on the above settings, the equilibrium length of each elastic bond between consecutive nodal points is
\begin{equation}\label{eq:l_0}
\ell_0=\sqrt{2\sigma_0^2\left(1-\cos\\psi_0\right)+b^2}\simeq0.705\text{nm}
\end{equation}
and the equilibrium bending angle at each nodal point is
\begin{equation}\label{eq:theta_0}
\theta_0=\cos^{-1}\left[\dfrac{2\sigma_0^2\left(1-\cos\psi_0\right)\cos\psi_0+b^2}{2\sigma_0^2\left(1-\cos\psi_0\right)+b^2}\right]\simeq31.4\degree
\end{equation}

\subsection{Energy functions}
Here I introduce the essential energy functions for general double-stranded helical chains.
They are bonding and bending energy of the P- and Q-chain.
The total bonding energy is given by
\begin{equation}\label{eq:bond}
V_{bond}=\sum_{i=1}^{N-1}\dfrac{1}{2}k_{bond}\left(\left|\textbf{P}_{i+1}-\textbf{P}_{i}\right|-\ell_0\right)^2+\sum_{i=1}^{N-1}\dfrac{1}{2}k_{bond}\left(\left|\textbf{Q}_{i+1}-\textbf{Q}_{i}\right|-\ell_0\right)^2
\end{equation}
where $\ell_0$ has been determined in Eq.~\ref{eq:l_0}.
$k_{bond}$ is the bonding rigidity.
The total bending energy is given by
\begin{equation}\label{eq:bend}
V_{bend}=\sum_{i=2}^{N-1}\dfrac{1}{2}k_{bend}\left(\theta_i^{(P)}-\theta_0\right)^2+\sum_{i=2}^{N-1}\dfrac{1}{2}k_{bend}\left(\theta_i^{(Q)}-\theta_0\right)^2
\end{equation}
where $\theta_0$ has been determined in Eq.~\ref{eq:theta_0} and $k_{bend}$ is the bending rigidity.
$\theta_i^{(P)}$ and $\theta_i^{(Q)}$ are the bending angles in the P- and Q-chains respectively and are given by~\cite{old}
\begin{equation}
\begin{split}
\theta_i^{(P)} & =\cos^{-1}\left(\dfrac{\textbf{P}_{i}-\textbf{P}_{i-1}}{\left|\textbf{P}_{i}-\textbf{P}_{i-1}\right|}\cdot\dfrac{\textbf{P}_{i+1}-\textbf{P}_{i}}{\left|\textbf{P}_{i+1}-\textbf{P}_{i}\right|}\right)\qquad \left(i=2,\dots,N-1\right) \\
\theta_i^{(Q)} & =\cos^{-1}\left(\dfrac{\textbf{Q}_{i}-\textbf{Q}_{i-1}}{\left|\textbf{Q}_{i}-\textbf{Q}_{i-1}\right|}\cdot\dfrac{\textbf{Q}_{i+1}-\textbf{Q}_{i}}{\left|\textbf{Q}_{i+1}-\textbf{Q}_{i}\right|}\right)\qquad \left(i=2,\dots,N-1\right)
\end{split}
\end{equation}

The two rigidity parameters can not be found directly by experiments.
To evaluate them, they take experimental data and assume that the energy is a quadratic function of the bending angles and set the parameters since they reproduce as better as they can the quadratic function.
The estimated parameters are
\begin{equation}
k_{bond}=1000\text{pN/np}\qquad k_{bend}=7000\text{pNnm}
\end{equation}

\subsection{Monte Carlo algorithm}
They use the Metropolis Monte Carlo algorithm to incorporate the effects of thermal and noisy environments.
Suppose that we have a ''current'' conformation of the DNA model with total energy $V$.
Then, we randomly choose one angle between all the bending ($\Theta_i$), dihedral ($\Phi_i$) and twist ($\psi_i$) angles and perturb it by $\pm1\degree$ randomly to obtain a ''trial'' conformation with energy $V_{tr}$.
If the trial conformation has a lower energy ($V_{tr}<V$), the trial conformation is accepted.
If the trial conformation has a higher energy ($V_{tr}>V$), we accept the trial conformation with the probability
\begin{equation}\label{eq:prob}
p=\exp{\left(-\dfrac{V_{tr}-V}{k_{B}T}\right)}
\end{equation}
with temperature set to $T=$\SI{300}{\kelvin}

Note that the Monte Carlo algorithm of the present study uses only the three kinds of angles, $\Theta$, $\Phi$, and $\psi$, as mentioned above for making trial movements.
This makes it possible to highlight the roles of the coupling between bending and writhing of the DNA model.
On the other hand, however, this algorithm can limit the movements of other degrees of freedom of the DNA model.

\section{Results}
\subsection{Asymmetric bend-writhe elasticity of the DNA model}\label{sec:bend_writh}
In this section I study the fundamental elasticity of the DNA model with particular attention to the coupling between bending and writhing of the central backbone.
Fig.~\ref{fig:bend}(a)-(c) show the conformations of the DNA model. where all dihedral angles are set to $\Phi_i=-4\degree$ and all bending angles are set to Fig.~\ref{fig:bend_a} $\Theta_i=1\degree$, Fig.~\ref{fig:bend_b} $\Theta_i=3\degree$, and Fig.~\ref{fig:bend_c} $\Theta_i=5\degree$, respectively.
Fig.~\ref{fig:bend}(d)-(f) show the same bending angles but dihedral angles are set to $\Phi_i=+4\degree$.
From these figures, we confirm that negative dihedral angles $\Phi_i$ give rise to left-handed super-helical conformations, while positive dihedral angles $\Phi_i$ give rise to right-handed super-helical conformations.
We also see that the more bending angles $\Theta_i$ increase, the more super-helical pitch of the DNA model decreases and super-helical radius increases, under conditions of fixed dihedral angles $\Phi_i$.

Fig.~\ref{fig:bend_energy} shows the dependence of the total energy of the DNA mode with $N=100$ nodes, $V_{DNA}=V_{bond}+V_{bend}$ defined by Eq.~\ref{eq:bond} and Eq.~\ref{eq:bend} on writhing the central backbone.
All the dihedral angles are set to the same value $\Phi$ that changes from $-4\degree$ to $4\degree$, while all bending angles are fixed to $1\degree$ (solid curve, blue), $3\degree$ (broken curve, green), and $5\degree$ (dash-dotted curve, red).
There is an evident asymmetry between left-handed writhing ($\Phi<0$) and right-handed writhing ($\Phi>0$).
In particular left-handed writhing has lower energy value at the same absolute value of dihedral angle, and increasing bending angles this difference increases.
This result clearly indicates the intrinsic asymmetric coupling between bending and writhing of the DNA model.
Specifically, the DNA model has a preference for left-handed writhing upon bending.
The asymmetric bend-writhe elasticity observed in Fig.??? should be the result of the right-handedness of the double-stranded model of DNA since there is no other chirality in the model or in the potential energy functions of the DNA model.

\begin{figure}[htbp]
\centering
\begin{subfigure}{0.3\textwidth}
\includegraphics[width=\textwidth]{-4_1.pdf}
\caption{$\Phi=-4\degree\ \Theta=1\degree$}
\label{fig:bend_a}
\end{subfigure}
\begin{subfigure}{0.3\textwidth}
\includegraphics[width=\textwidth]{-4_3.pdf}
\caption{$\Phi=-4\degree\ \Theta=3\degree$}
\label{fig:bend_b}
\end{subfigure}
\begin{subfigure}{0.3\textwidth}
\includegraphics[width=\textwidth]{-4_5.pdf}
\caption{$\Phi=-4\degree\ \Theta=5\degree$}
\label{fig:bend_c}
\end{subfigure}
\begin{subfigure}{0.3\textwidth}
\includegraphics[width=\textwidth]{4_1.pdf}
\caption{$\Phi=4\degree\ \Theta=1\degree$}
\label{fig:bend_d}
\end{subfigure}
\begin{subfigure}{0.3\textwidth}
\includegraphics[width=\textwidth]{4_3.pdf}
\caption{$\Phi=4\degree\ \Theta=3\degree$}
\label{fig:bend_e}
\end{subfigure}
\begin{subfigure}{0.3\textwidth}
\includegraphics[width=\textwidth]{4_5.pdf}
\caption{$\Phi=4\degree\ \Theta=5\degree$}
\label{fig:bend_f}
\end{subfigure}
\caption{Conformation to the DNA model with $N=100$ base pairs for all the combination between dihedral angles ($\Phi=\pm 4\degree$) and bending angles ($\Theta=$\SIlist{1;3;5}{\degree}).}
\label{fig:bend}
\end{figure}

\begin{figure}[htbp]
\centering
\includegraphics[width=.5\textwidth]{energy.pdf}
\caption{Dependence of total energy of the DNA model with $N=100$ nodes, on the writhe of the central backbone, i.e., the value of dihedral angles. Each line is the total energy at different bending angles. All dihedral and bending angles are equal for all the nodes.}
\label{fig:bend_energy}
\end{figure}

\subsection{Chiral selection in wrapping of the DNA model around a spherical core particle}\label{sec:core}
The asymmetric bend-writhe elasticity observed in Sec.~\ref{sec:bend_writh} indicates that DNA may have a preference on the directionality, i.e., chirality, in wrapping around a core particle.
This is a useful test because in a nucleosome, DNA usually wraps around a protein core particle called histone octamer about 1.75 times in a left-handed manner~\cite{1}\cite{2}.
In this section I try to account for the stability of hte left-handed wrapping of DNA around a core particle in terms of the asymmetric elasticity of the DNA model.

In the Monte Carlo simulation I use a Morse potential instead of electrostatic one for the attraction between DNA and core particle, since it requires less machine time.
The potential is
\begin{equation}\label{eq:core}
V_{core}=\sum_{i=1}^{N}D_{core}\left\{\exp\left[-2\beta_{core}\left(\left|\textbf{r}_{i}-\textbf{r}_{core}\right|-\sigma_{core}\right)\right]-2\exp\left[-\beta_{core}\left(\left|\textbf{r}_{i}-\textbf{r}_{core}\right|-\sigma_{core}\right)\right]\right\}
\end{equation}
where $\textbf{r}_i$ is the position of the i-th nodal point of the central backbone and $\textbf{r}_{core}$ is the position of the center of the spherical core.
$\sigma_{core}=$\SI{4.5}{\nm} is the equilibrium distance between the center of the core and each nodal point of the central backbone, $\beta_{core}=$\SI{2.0}{\per\nm} determines the width of the Morse potential, and $D_{core}=10$pNnm determines the strength of the interaction.

To avoid overlaps of the DNA with itself, they introduce a potential for the excluded volume effect (it is the repulsive part of the Morse potential)
\begin{equation}\label{eq:exc}
V_{exc}=\sum_{i=1}^{N-n}\sum_{j=i+n}^{N}D_{exc}\exp\left[-2\beta_{exc}\left(\left|\textbf{r}_{i}-\textbf{r}_{j}\right|-\sigma_{exc}\right)\right]
\end{equation}
where they set $\beta_{exc}=$\SI{2.0}{\per\nm}, $\sigma_{exc}=$\SI{2.1}{\nm}, and $D_{exc}=1.0$pNnm.
The parameter $n=7$ removes the repulsions between very close nodal points in comparison to $\sigma_{exc}$.

I perform Monte Carlo simulation for a DNA model with $N=200$ nodes, where the position of the first two base pairs and of the core are fixed.
The total energy function used to perform the Metropolis algorithm with Eq.~\ref{eq:prob} now is
\begin{equation}\label{eq:hist_energy}
V=V_{bond}+V_{bend}+V_{core}+V_{exc}
\end{equation}

Fig.~\ref{fig:hist} show some of the results of Monte Carlo simulations, where $\textbf{r}_{core}=\left(0,\ -4.5,\ 10.2\right)^T$ in unit of \si{\nm}.
We see DNA wrapping around the histone in a left-handed manner up to about $2$ turns.

Fig.~\ref{fig:hist_en} shows the evolution of the total energy: we see as energy tends to decrease and equilibrate as the DNA wraps around the core, except for the initial abrupt increase.
Fig.~\ref{fig:hist_wr} shows the corresponding evolution of a wrapping number~\cite{very_old} defined by
\begin{equation}\label{eq:wrap_num}
W=\dfrac{b\left(N_{ad}-1\right)}{2\pi\sigma_{core}}
\end{equation}
where $N_{ad}$ is the number of nodal points of the DNA model adsorbed on the core; they assume that a node is adsorbed if satisfies $\left|\textbf{r}_i-\textbf{r}_{core}\right|\leq$ \SI{4.8}{\nm}.
We see that the wrapping number increases as DNA wraps and equilibrates at around $W=1.75$.
In order to judge the chirality of the wrapping in a systematic manner, they utilize a chirality parameter~\cite{very_old} defined by
\begin{equation}\label{eq:chirality}
C=\langle\textbf{m}_{i}\rangle_{ad}\cdot\left(\langle\textbf{r}_{i}\rangle_{tail}-\langle\textbf{r}_{i}\rangle_{head}\right)
\end{equation}
where $\langle\textbf{m}_{i}\rangle_{ad}$ is obtained by averaging the vectors $\textbf{m}_i=\left(\textbf{r}_{i}-\textbf{r}_{i-1}\right)\times\left(\textbf{r}_{i+1}-\textbf{r}_{i}\right)$ over all adsorbed nodes and then normalizing it.
$\langle\textbf{r}_{i}\rangle_{head}$ represents the average of position vectors of the first half (with smaller subscripts $i$)of the adsorbed nodal points of the central backbone of DNA, whereas $\langle\textbf{r}_{i}\rangle_{tail}$ represents the average of position vectors of the last half (with larger subscripts $i$) of the adsorbed nodal points of the central backbone.
Positive sign of $C$signifies right-handed wrapping while negative sign of $C$ signifies left-handed wrapping.
Fig.~\ref{fig:hist_ch} shows the evolution of the chirality parameter, and we see that $C$ takes mostly negative values.
Fig.~\ref{fig:hist_ch_pr} is the probability distribution of $C$ obtained from $10$ runs of Monte Carlo simulations, each of which consists of $6\times 10^5$ steps.
In order to avoid the bias due to the specific initial conditions introduced above, we have discarded the data of initial $3\times10^5$ steps from each Monte Carlo simulation.
The clear peak at negative value of $C$ indicates the strong propensity of the DNA model to wrap around the core in the left-handed manner.
The only negative values of $C$ indicates that in all the simulations the DNA wraps in left-handed manner around core particle.

\begin{figure}[htbp]
\centering
\begin{subfigure}{.49\textwidth}
\includegraphics[width=\textwidth]{hist_0.pdf}
\caption{0th step}
\label{fig:hist_a}
\end{subfigure}
\begin{subfigure}{.49\textwidth}
\includegraphics[width=\textwidth]{hist_1.pdf}
\caption{100000th step}
\label{fig:hist_b}
\end{subfigure}
\begin{subfigure}{.49\textwidth}
\includegraphics[width=\textwidth]{hist_3.pdf}
\caption{300000th step}
\label{fig:hist_c}
\end{subfigure}
\begin{subfigure}{.49\textwidth}
\includegraphics[width=\textwidth]{hist_6.pdf}
\caption{600000th step}
\label{fig:hist_d}
\end{subfigure}
\caption{Monte Carlo simulation for wrapping of the DNA model with $N=200$ nodes around a spherical core particle at \SIlist{0;100000;300000;600000}{} step.}
\label{fig:hist}
\end{figure}

\begin{figure}[htbp]
\centering
\begin{subfigure}{.49\textwidth}
\includegraphics[width=\textwidth]{hist_en.png}
\caption{}
\label{fig:hist_en}
\end{subfigure}
\begin{subfigure}{.49\textwidth}
\includegraphics[width=\textwidth]{hist_wr.png}
\caption{}
\label{fig:hist_wr}
\end{subfigure}
\begin{subfigure}{.49\textwidth}
\includegraphics[width=\textwidth]{hist_ch.png}
\caption{}
\label{fig:hist_ch}
\end{subfigure}
\begin{subfigure}{.49\textwidth}
\includegraphics[width=\textwidth]{hist_ch_pr.pdf}
\caption{}
\label{fig:hist_ch_pr}
\end{subfigure}
\caption{(a) the evolution of total energy(Eq.~\ref{eq:hist_energy}).
(b) the evolution of wrapping number (Eq.~\ref{eq:wrap_num}).
(c) the evolution of chirality parameter (Eq.~\ref{eq:chirality}).
(d) probability distribution of the chirality parameter obtained from $10$ runs of Monte Carlo simulation of $6\times 10^5$ steps without the first $3\times 10^5$.}
\label{fig:hist_params}
\end{figure}

\subsection{Chirality of the DNA model adsorbed on a uniform rod}
Chiral selectivity of the DNA model observed in Sec.~\ref{sec:core} indicates that DNA may also exhibit chiral selectivity in wrapping around a uniform rod.
Indeed, conformation of DNA around rod-like molecules, such as carbon nanotube~\cite{rod_2}, is an interesting issue.
While wrapping of single-stranded DNA around carbon nanotubes is frequently studied, where wrapping geometry is known to be very sensitive to the DNA sequence~\cite{rod_1}, we study here the chirality of the double-stranded DNA model adsorbed on a hypothetical uniform rod.

I fix the uniform rod so that the central axis of the rod coincides with the z-axis of the space.
They then assume an attractive interaction between DNA and the uniform rod using the Morse potential as
\begin{equation}\label{eq:rod_1}
V_{rod 1}=\sum_{i=1}^{N}D_{rod}\left\{\exp\left[-2\beta_{rod}\left(\sqrt{x_{i}^{2}+y_{i}^{2}}-\sigma_{rod}\right)\right]-2\exp\left[-\beta_{rod}\left(\sqrt{x_{i}^{2}+y_{i}^{2}}-\sigma_{rod}\right)\right]\right\}
\end{equation}
where $x_{i}$ and $y_{i}$ represent x- and y-components of the position vector of the central backbone of the DNA model.
$\sigma_{rod}$ determines the effective radius of the rod including the DNA radius.
I will examine three different values: \SIlist{2.1;3.1;4.1}{\nm}.
The parameter $D_{rod}=20$pNnm determines the strength of the interaction between rod and DNA, while $\beta_{rod}=$\SI{2.0}{\per\nm} determines the width of the Morse potential.
Here I use a DNA model with $N=100$ and introduce a excluded volume effect exactly the same as in Eq.~\ref{eq:exc}.
The total energy used to calculate probability (Eq.~\ref{eq:prob}) in the Monte Carlo simulations is
\begin{equation}\label{eq:rod_1_energy}
V=V_{bond}+V_{bend}+V_{rod 1}+V_{exc}
\end{equation}
The first nodal point of the central backbone is fixed to the space at $\left(\sigma_{rod},\ 0,\ 0\right )^T$.
The straight equilibrium conformation is thermally perturbed to take superhelical or undulating conformations around the rod.
They introduce a wrapping number to characterize chirality and the number of wrapping
\begin{equation}\label{eq:wrap_rod}
W_{rod}=\dfrac{1}{2\pi}\sum_{i=2}^{N-1} sgn\left [\hat{\textbf{z}}\cdot\textbf{n}_{i}\right ]\cos^{-1}\left (\dfrac{x_{i}x_{i+1}+y_{i}y_{i+1}}{\sqrt{x_{i}^2+y_{i}^2}\sqrt{x_{i+1}^2+y_{i+1}^2}}\right )
\end{equation}
where $\hat{\textbf{z}}=\left (0,\ 0,\ 1\right )^T$, $\textbf{n}_i$ is a unit vector perpendicular to the central backbone at $i$-th node defined by  $\textbf{n}_i=\left (\textbf{r}_{i}-\textbf{r}_{i-1}\right )\times\left (\textbf{r}_{i+1}-\textbf{r}_{i}\right )/\left |\left (\textbf{r}_{i}-\textbf{r}_{i-1}\right )\times\left (\textbf{r}_{i+1}-\textbf{r}_{i}\right )\right |$, and $sgn\left [\right ]$ is the sign function.
Positive sign of $W_{rod}$ means right-handed wrapping of DNA around the rod, while negative sign means left-handed wrapping.
Absolute value of $W_{rod}$ can be regarded as a measure of the number of wrapping turns of the DNA model around the rod.

\subsection{Chiral selection in crossover and braiding of two juxtaposed DNA molecules}

\subsection{Dependence of braiding chirality on the strength of DNA-DNA interaction}

\section{Conclusions}

\clearpage
\addcontentsline{toc}{section}{\refname}
\printbibliography
\end{document}
